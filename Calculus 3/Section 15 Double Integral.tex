\documentclass{article}

\usepackage{amsmath, amsthm, amssymb, amsfonts}
\usepackage{thmtools}
\usepackage{graphicx}
\usepackage{setspace}
\usepackage{geometry}
\usepackage{float}
\usepackage{hyperref}
\usepackage[utf8]{inputenc}
\usepackage[english]{babel}
\usepackage{framed}
\usepackage[dvipsnames]{xcolor}
\usepackage{tcolorbox}

\colorlet{LightGray}{White!90!Periwinkle}
\colorlet{LightOrange}{Orange!15}
\colorlet{LightGreen}{Green!15}
\colorlet{LightYellow}{Yellow!15}

\newcommand{\HRule}[1]{\rule{\linewidth}{#1}}

\declaretheoremstyle[name=Theorem,]{thmsty}
\declaretheorem[style=thmsty,numberwithin=section]{theorem}
\tcolorboxenvironment{theorem}{colback=LightGray}

\declaretheoremstyle[name=Definition,]{thmsty}
\declaretheorem[style=thmsty,numberwithin=section]{definition}
\tcolorboxenvironment{definition}{colback=LightYellow}

\declaretheoremstyle[name=Proposition,]{prosty}
\declaretheorem[style=prosty,numberlike=theorem]{proposition}
\tcolorboxenvironment{proposition}{colback=LightOrange}

\declaretheoremstyle[name=Principle,]{prcpsty}
\declaretheorem[style=prcpsty,numberlike=theorem]{principle}
\tcolorboxenvironment{principle}{colback=LightGreen}

\setstretch{1.2}
\geometry{
    textheight=9in,
    textwidth=5.5in,
    top=1in,
    headheight=12pt,
    headsep=25pt,
    footskip=30pt
}

% ------------------------------------------------------------------------------

\begin{document}

% ------------------------------------------------------------------------------
% Cover Page and ToC
% ------------------------------------------------------------------------------

\title{ \normalsize \textsc{}
		\\ [2.0cm]
		\HRule{1.5pt} \\
		\LARGE \textbf{\uppercase{Section 15}
		\HRule{2.0pt} \\ [0.6cm] \LARGE{Double Integrals} \vspace*{10\baselineskip}}
		}
\date{}
\author{\textbf{Author}\\
Jonathan Ruiz}

\maketitle
\newpage

\tableofcontents
\newpage

% ------------------------------------------------------------------------------

\section{Double Integrals Over Rectangles}
\subsection{Intro and Riemann Sum}
For a curve $y = f(x) > 0$ on $[a, b]$ we approximated the area under the curve using rectangles

\begin{enumerate}
    \item Divide the interval $\Delta x$ = $\cfrac{b-a}{n}$
    \item Height = $f(x_i ^ *)$ where $x_i ^ *$ is a sample on the ith subinterval
    \item Area of each rectangle = $f(x_i ^ *) \Delta x$
    \item Add up the area: $A \approx \sum_{i=1}^{n} f(x_i ^ *) \Delta x$
\end{enumerate}

Exact Area: $A = \lim_{n\to\infty} \sum_{i=1}^{n} f(x_i ^ *)$\\
\begin{definition}

\textbf{Definition of Definite Integral:}

$\int_{a}^{b} f(x) dx = \lim_{n\to\infty} \sum_{i=1}^{n} f(x_i ^ *)$
\end{definition}



Find the volume of a solid below a surface $z=f(x, y) \geq 0$ over the rectangular region:
$$R = [a, b] \times [c, d] = \{ (x, y) \in R^2 | a \leq x \leq b, c \leq y \leq d \}$$

We can approximate this volume by dividing $[a,b] \times [c,d]$ into sub-regions 

$$R_{ij} = [x_{i-1}, x_{i}] \times [y_{j-1}, y_j]$$

Each sub-region has area $\Delta x \Delta y$ where $\Delta x = \cfrac{b-a}{m}$ and $\Delta y = \cfrac{d - c}{n}$. 
Height is $f(x_i^*, y_j^*)$.

The volume of each column = $f(x_i^*, y_j^*)\Delta x \Delta y$ and to add the volume of all the columns

we can do 
$$V = \sum_{i=1}^m \sum_{j=1}^n f(x_i^*, y_j^*) \Delta A$$

Where $\Delta A = \Delta x \Delta y$

\begin{definition}
    \textbf{Double Integral of $f$ Over Rectangle $R$}

    $\iint_{R} f(x, y) dA = \lim_{m,n\to\infty} \sum_{i =1}^m \sum_{j=1}^n f(x_i^*, y_j^*) \Delta A$
    if the limit exists
\end{definition}

\begin{theorem}
    \textbf{Fubini's Theorem}
    If f is continuous on the rectangle 
    
    $R = \{(x, y) \mid a \leq x \leq b, c \leq y \leq d \}$
    $\\ \implies\iint_R f(x, y) dA = \int_a^b \int_c^d f(x, y) dydx = \int_c^d \int_a^b f(x, y) dxdy$\\
    
    More generally, this is true if we assume that f is bounded on R, f is discontinuous only on a finite number of smooth curves, and the iterated integrals exist.
\end{theorem}

\newpage

\begin{definition}
    \textbf{Average Value of a Region}

    $f_{avg} = \cfrac{1}{A(R)} \iint_R f(x, y) dA$
    where A(R) is the area of the region
\end{definition}
\section{Double Integrals Over General Regions}

\textbf{Two types of Regions D}

Type I: $D = \{(x, y) \mid a \leq x \leq b, g_1(x) \leq y \leq g_2(x) \}$

Type II: $D = \{(x, y) \mid a \leq y \leq b, h_1(y) \leq x \leq h_2(y)\}$

If f is continuous on a type I region, then 
$\iint_D f(x, y)dA = \int_a^b \int_{g_1(x)}^{g_2(x)} f(x, y) dydx$


If f is continuous on a type II region, then 
$\iint_D f(x, y) dA = \int_c^d \int_{h_1(y)}^{h_2(y)} f(x, y) dxdy$\\


\textbf{Properties of Double Integrals}
\begin{enumerate}
\item $\iint_D [f(x, y) + g(x, y)] dA = \iint_D f(x, y) dA + \iint_D g(x, y)dA$
\item $\iint_D cf(x, y) dA = c\iint_D f(x,y) dA$  where c is constant
\item If $f(x, y) \geq g(x, y)$ for all $(x, y) \in D$, then $\iint_D f(x, y) dA \geq \iint_D g(x, y) dA$
\item  $\iint_D f(x, y) dA = \iint_{D_1} f(x, y) dA + \iint_{D_2} f(x, y) dA$
\item If $m \leq f(x, y) \leq M$ for all $(x,y) \in D_1$ then $m A(D) \leq \iint_D f(x, y) dA \leq M A(D)$
\end{enumerate}

\section{Double Integrals in Polar Coordinates}

%\begin{proposition}
%    This is a proposition.
%\end{proposition}

%\begin{principle}
%    This is a principle.
%\end{principle}


% Maybe I need to add one more part: Examples.
% Set style and colour later.

%\subsection{Pictures}

%\begin{figure}[htbp]
%    \center
%    \includegraphics[scale=0.06]{img/photo.jpg}
%    \caption{Sydney, NSW}
%\end{figure}

%\subsection{Citation}

%This is a citation\cite{Eg}.

\newpage

% ------------------------------------------------------------------------------
% Reference and Cited Works
% ------------------------------------------------------------------------------

\bibliographystyle{IEEEtran}
\bibliography{References.bib}

% ------------------------------------------------------------------------------

\end{document}